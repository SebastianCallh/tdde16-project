\chapter{Data}
The data was provided by a competition on the website Kaggle for
identifying various types of comments~\cite{toxic-kaggle}. It contains
Wikipedia comments in plain text together with labels indicating toxic
behaviour.

\begin{table}[H]
  \centering
  \caption{Example comments with their labels.}    
  \label{tbl:comments}
  \resizebox{12cm}{!}{%
\begin{tabular}{l|ccccccc}
comment text &  toxic &  severe toxic &  obscene &  threat &  insult
  &  identity hate \\
  \hline
  "\textbackslash n Sure, but the lead must briefly summarize ... &      0 &             0 &        0 &       0 &       0 &              0 \\
TFD \textbackslash n\textbackslash nI think we just eced. I think we respo... &      0 &             0 &        0 &       0 &       0 &              0 \\
 You are gay or antisemmitian? \textbackslash n\textbackslash nArchangel WH... &      1 &             0 &        1 &       0 &       1 &              1 \\
           FUCK YOUR FILTHY MOTHER IN THE ASS, DRY! &      1 &             0 &        1 &       0 &       1 &              0 \\
  I'm Sorry \textbackslash n\textbackslash nI'm sorry I screwed around with ... &      1 &             0 &        0 &       0 &       0 &              0 \\
  I don't believe the Lisak criticism present th... &      0 &             0 &        0 &       0 &       0 &              0 \\
  You had a point, and it's now ammended with ap... &      0 &             0 &        0 &       0 &       0 &              0 \\
  In other words, you're too lazy to actually po... &      0 &             0 &        0 &       0 &       0 &              0 \\
  "\textbackslash nAs for your claims of ""stalking"", that is... &      0 &             0 &        0 &       0 &       0 &              0 \\
  "::::Jmabel; in regards to predominant scholar... &      0 &             0 &        0 &       0 &       0 &              0 \\
\end{tabular}
  }
\end{table}
The different labels of the comments were ``toxic'', ``severe toxic'' ,
``obscene'', ``threat'', ``insult'', and ``identity hate''. Some
truncated comments and their labels can be seen in
Table~\ref{tbl:comments}. 
\begin{figure}[H]
  \centering
  \includegraphics[width=0.8\textwidth]{graphics/label-dist}
  \caption{The count of different labels. Since labels are independent,
    one comment can contribute to several label counts.}\label{fig:labels-dist}
\end{figure}
The data was skewed towards unflagged comments. Figure~\ref{fig:labels-dist} show the counts of the
different labels, where it can be seen that the labels ``severe
toxic'', ``threat'' and ``identity hate'' are increadibly uncommon,
for instance.
\begin{figure}[H]
  \centering
  \includegraphics[width=0.8\textwidth]{graphics/comment-word-count}
  \caption{Density plot of comment word counts.}\label{fig:comment-word-count}
\end{figure}
Plotting the word count of the comments show that almost all comments
have less than 200 words in them, as shown in Figure~\ref{fig:comment-word-count}.